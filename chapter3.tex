\chapter{مروری بر کارهای مشابه}
 این فصل به معرفی و بررسی کلی محصولات با کارایی مشابه اختصاص یافته‌است.

\section{کارهای مشابه}
با توجه به محبوبیت رایانش ابری و خدمات ارائه زیرساخت، اکثر قریب به اتفاق شرکت‌های فعال در این حوزه اقدام به ارائه درگاه‌های ارائه خدمات ابری به صورت رابط برنامه‌نویسی کرده اند که در ادامه با این محصولات آشنا می‌شویم.

\subsection{نمونه‌های داخلی}
از بین شرکت‌های داخلی فعال در این حوزه، شرکت ابرآروان\LTRfootnote{http://arvancloud.ir} اقدام به ارائه‌ی خدمات \lr{IaaS} در قالب رابط ارتباطی تحت وب کرده است.

این رابط که در این آدرس\LTRfootnote{https://www.arvancloud.ir/docs/api/iaas/} تفسیر شده است، تمامی خدمات قابل تعریف برروی ماشین مجازی و منابع مجازی را مشخص کرده. با توجه به متن‌بسته بودن این سرویس، نمی‌توان بیش از این در مورد معماری و جزئیات پیاده‌سازی خدماتی که ارائه می‌شود، صحبت کرد.


یکی دیگر از شرکت‌های ارائه دهنده خدمات مشابه، شرکت آسیاتک\LTRfootnote{https://asiatech.cloud} است. مستندات مربوط به عملکرد سرویس ارائه خدمات ابری این شرکت، در این آدرس\LTRfootnote{https://asiatech.cloud/api-doc/} قرارداده شده.


\subsection{نمونه‌های خارجی}
شرکت‌های بزرگ فعال در حوزه‌ی خدمات ابری نظیر گوگل، مایکروسافت، آمازون، دیجیتال اوشن، اوراکل و آی‌بی‌ام، همگی سرویس ارائه‌ی خدمات \lr{IaaS} را به صورت عمومی برای استفاده کاربران فراهم کرده‌اند. معماری و نحوه‌ی سرویس‌دهی تمامی این ارائه دهندگان به شکل مشابه و بر مبانی الگوی پرداخت بر اساس مصرف\LTRfootnote{Pay-As-You-Go} است. منتها به دلیل متن‌بسته بودن این سرویس‌ها، اطلاعات بیشتری در مورد جزئیات پیاده‌سازی آن‌ها در دسترس نیست. با این وجود، با بررسی دقیق‌تر و جزئی‌تر مستندات سرویس‌ها می‌توان کلیاتی در مورد طراحی و عملکرد سامانه‌ها به دست آورد که در فصل بعد به بررسی این موارد می‌پردازیم.

مستندات مربوط به ارائه‌ی خدمات \lr{IaaS} این توسعه دهندگان در پیوست این گزارش قرار داده شده.