%% -!TEX root = AUTthesis.tex
% در این فایل، عنوان پایان‌نامه، مشخصات خود، متن تقدیمی‌، ستایش، سپاس‌گزاری و چکیده پایان‌نامه را به فارسی، وارد کنید.
% توجه داشته باشید که جدول حاوی مشخصات پروژه/پایان‌نامه/رساله و همچنین، مشخصات داخل آن، به طور خودکار، درج می‌شود.
%%%%%%%%%%%%%%%%%%%%%%%%%%%%%%%%%%%%
% دانشکده، آموزشکده و یا پژوهشکده  خود را وارد کنید

\faculty{دانشکده مهندسی کامپیوتر}
% گرایش و گروه آموزشی خود را وارد کنید
\department{}
% عنوان پایان‌نامه را وارد کنید
\fatitle{رابط تحت وب خدمات زیرساخت به عنوان خدمت
\\[.75 cm]
}
% نام استاد(ان) راهنما را وارد کنید
\firstsupervisor{دکتر محمود ممتازپور}
%\secondsupervisor{استاد راهنمای دوم}
% نام استاد(دان) مشاور را وارد کنید. چنانچه استاد مشاور ندارید، دستور پایین را غیرفعال کنید.
% \firstadvisor{دکتر مهدی همایون‌پور}
%\secondadvisor{استاد مشاور دوم}
% نام نویسنده را وارد کنید
\name{سیدمحمد}
% نام خانوادگی نویسنده را وارد کنید
\surname{فاطمی}
%%%%%%%%%%%%%%%%%%%%%%%%%%%%%%%%%%
\thesisdate{اسفند ۱۴۰۱}

% چکیده پایان‌نامه را وارد کنید
\fa-abstract{
امروزه الگوی مصرف منابع محاسباتی و شبکه‌ای بسیار به الگوی رایانش ابری نزدیک شده. با این وجود ارائه دهندگان خدمات زیرساخت در داخل کشور همچنان فاصله زیادی تا ارائه خدمات منطبق بر الگوی رایانش ابری دارند. یکی از خدمات معروف مورد نیاز کاربران، نیاز به زیرساخت به عنوان خدمت است. چالش پیاده‌سازی این خدمت، پیاده‌سازی زیرساخت و درگاه‌های مورد نیاز جهت تحویل خدمات به کاربر است. در مورد مسئله اول، راه‌حل‌های متن‌باز و صنعتی متعددی ارائه شده‌اند. ایراد استفاده مستقیم از این راه‌حل‌های بدون نیاز به تغییر و پیاده‌سازی درگاه ارتباطی خصوصی، عدم امکان شخصی‌سازی امکانات و نحوه ارائه خدمات است. به علت متفاوت بودن الگو‌های ارائه‌ی خدمت و مدیریت کاربران، این ابزار‌ها، پیاده‌سازی این قسمت از سامانه‌ را برعهده‌ی ‌کاربر استفاده کننده قرار داده‌اند. در این پروژه به بررسی نیازمندی‌های یک سامانه‌ی ارائه‌ی خدمات زیرساخت به عنوان خدمت پرداخته شده و با بررسی و مشخص کردن نیازمندی‌ها و تکنولوژی‌ها، یک راه‌حل جهت پیاده‌سازی این سرویس بر روی زیرساخت \lr{VMWare Cloud Director} ارائه شده. در مراحل مختلف طراحی، تصمیمات و گزینه‌های مختلف مطرح شده و دلیل تصمیم گیری‌ها و انتخاب تکنولوژی شرح داده شده. پس از پیاده‌سازی، با طرح و اجرای ارزیابی‌های مختلف بر روی این راه‌حل، از صحت عملکرد آن در سناریو‌های واقعی اطمینان حاصل پیدا شده. در انتها با بررسی کلیت سامانه، نقاط بهبود و جایگاه‌های قابل توسعه در پروژه شرح داده‌شده که بتوان در جهت‌های مختلف این پروژه را توسعه داد و خدمات جامع‌تری را پوشش داد.
}
%توجه: ‌در اعداد اعشاری قسمت صحیح و اعشار جا به جا باید نوشته شوند تا در متن به طور صحیح نمایش داده شوند. به عنوان مثال در متن بالا عدد ۲۹.۰۷ را جا به جا وارد کرده تا در متن به طور صحیح نمایش داده شود. 

% کلمات کلیدی پایان‌نامه را وارد کنید
\keywords{رایانش ابری، زیرساخت به عنوان خدمت، رابط برنامه‌نویسی تحت وب، معماری میکروسرویس}


\AUTtitle
%%%%%%%%%%%%%%%%%%%%%%%%%%%%%%%%%%
\vspace*{7cm}
\thispagestyle{empty}
\begin{center}
\includegraphics[height=5cm,width=12cm]{besm}
\end{center}