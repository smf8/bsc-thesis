\chapter{پیاده‌سازی سیستم}
در این فصل، به مراحلی که برای پیاده‌سازی این سیستم طی شده‌است، پرداخته شده‌است.

\section{نیازمندی‌های سامانه}
هدف ما، پیاده‌سازی یک سامانه‌ی کاربردی کامل است که با استفاده از رابط برنامه‌نویسی تحت وب، بتوانیم خدمات را از آن دریافت کنیم. این خدمات به شرح زیر می‌باشند:

\begin{enumerate}
\item \textbf{احراز هویت}: کاربران باید بتوانند با استفاده از نام کاربری و کلمه عبور احراز هویت شوند و سطوح دسترسی متفاوت داشته باشند.

\item \textbf{خدمات ابری}: کاربران باید بتوانند خدماتی که در حوزه‌ی \lr{IaaS} تعریف می‌شود را دریافت کنند.

\item \textbf{مدیریت و حسابداری}: کاربر مدیر سامانه باید بتواند مصرف زیرساخت‌ کاربران را به صورت دوره ای بررسی کند.
\end{enumerate}

این خدمات باید با دسترسی‌پذیری بالا\LTRfootnote{Highly Available} و به صورت توزیع‌شده\LTRfootnote{Distributed} توسعه پیدا کنند که پاسخگوی نیازمندی‌های کاربران در شرایط خاص و فشار‌های بسیار بالا باشد.

\section{طراحی و معماری سیستم}
با توجه به نیازمندی‌های مطرح شده برای خدمات سامانه، در ابتدا باید دامنه‌ی فعالیت‌های بخش‌های مختلف سامانه را مشخص کنیم. این تصمیم گیری باید به گونه‌ای باشد که بخش‌ها، کمترین وابستگی و جفت شدگی\LTRfootnote{Coupling} را داشته باشند.

با توجه به استقلال نیازمندی‌ها و عملکرد‌های ذکر شده برای سامانه، معماری کلی این سامانه، معماری میکروسرویس است. در این پیاده‌سازی، هر نیازمندی در یک سرویس پیاده‌سازی شده است، یک سرویس جهت انجام امور احراز هویت، یک سرویس برای تعامل با کاربران و تحویل درخواست‌های آنان، یک سرویس جهت تعامل با زیرساخت ابری جهت اجرای دستورات و در نهایت یک سرویس جهت امور نظارتی و مدیریتی تعریف شده که همه این سرویس‌ها در یک شبکه‌ی خصوصی و پشت سر یک ‌دروازه‌ی ورود رابط قرار گرفته. معماری کلی سامانه در شکل\ref{fig:30bird-arch} مشخص شده است. در ادامه‌ی این فصل جزئیات هر بخش از این معماری را دقیق‌تر بررسی می‌کنیم.

\begin{figure}
	\centering
	\includegraphics[scale=0.5]{figures/30bird-arch.png}
	\caption{معماری کلی سامانه}
	\label{fig:30bird-arch}
\end{figure}


\clearpage
\section{میکروسرویس‌ها}
کل سامانه از چهار میکروسرویس تشکیل شده: قفل، ناظم، خادم و باجه. هر میکروسرویس، مخصوص رسیدگی به امور بخشی از سامانه است. این میکروسرویس‌ها در قالب معماری کد تک مخزنه\LTRfootnote{Mono Repo} پیاده‌سازی شده‌اند و دارای اشتراک‌های زیادی در کد‌های پایه هستند.

در این بخش قصد داریم که این موارد مشترک را مطرح کنیم و سپس به بررسی جزئیات اختصاصی هر سرویس بپردازیم.

اجرای یک خدمت‌گزار وب\LTRfootnote{Web Server} توسط چهارچوب \lr{Echo} در زبان گو انجام می‌شود. نحوه عملکرد این چهارچوب به این شکل است که در ابتدا قوانین مربوط به مسیریاب\LTRfootnote{Router} تنظیم می‌شود، سپس برای هر مقصد\LTRfootnote{Endpoint}، یک رسیدگی کننده\LTRfootnote{Handler} تعریف می‌کنیم. به عنوان مثال برای عملیات ورود کاربر به سامانه، برای مسیر \lr{/api/login} یک رسیدگی کننده تعریف می‌کنیم که عملیات‌های لازم جهت ورود کاربر را اجرا می‌کند. این عملیات‌ها در این مثال شامل چک کردن ورودی درخواست، اتصال به پایگاه داده، چک کردن صحت اطلاعات فرستاده شده و ارسال یک کلید ورود برای کاربر است که همگی در قالب یک تابع به عنوان رسیدگی کننده‌ی این مقصد تعریف می‌شوند. همچنین در تمامی خدمت‌گزار‌های این سامانه، کاربران بایستی هویت خود را از طریق سربرگ‌های اجباری از پیش تعیین شده مشخص کنند و یک احراز هویت با کلید ثابت انجام دهند. این امر جهت بالا بردن امنیت دوچندان سامانه تعبیه شده‌است.
در ساختار کد این پروژه، ثبت این مقصدها به همراه توابع رسیدگی کننده در بسته \lr{api/http/handler} انجام می‌شود. جهت ثبت یک بسته جدید فقط کافیست که مشخصات مقصد و تابع رسیدگی کننده در فایلی داخل این بسته نوشته شود و با تابع \lr{RegisterEndpoints} مقصد مورد نظر داخل مسیریاب \lr{Echo} ثبت شود.

ایجاد برنامه‌های تحت خط‌فرمان\LTRfootnote{Command Line Application} توسط چهارچوب \lr{Cobra} پیاده‌سازی شده. این چهارچوب به برنامه‌نویسان این اختیار ‌را می‌دهد که برای برنامه‌های خود یک رابط خط فرمان\LTRfootnote{Command Line Interface} ایجاد کنند. با این کار، برنامه‌نویس می‌تواند با کد‌های نوشته شده برای پروژه، چندین فرمان قابل اجرا ایجاد کند. مزیت این‌کار در سناریو‌هایی که برنامه‌ی‌ نوشته شده نیاز به اجرای چند فرمان مختلف به طور همزمان در پس‌زمینه داشته باشد مشخص می‌شود.

مدیریت تنظیمات برنامه توسط چهارچوب \lr{Viper} انجام می‌شود. این چهارچوب، به کاربران این اجازه را می‌دهد که تنظیمات را در هر‌یک از ساختارهای معروف \lr{JSON}،\lr{YAML} و یا \lr{TOML} بنویسد و با قراردادن آن در یکی از آدرس‌های پیشفرض تنظیمات، آن‌ها را بر روی برنامه اعمال کند. ساختار انتخابی ما برای تنظیمات به صورت پیش‌فرض، ساختار \lr{YAML} است که در یک فایل هم‌اسم پروژه، در پوشه ریشه پروژه قرار گرفته.

تعامل و ارتباطات با پایگاه داده در اپلیکیشن‌ها با دو چهارچوب \lr{gorm} و \lr{ent} انجام می‌شود. مزیت چهارچوب \lr{gorm}، سادگی عملکرد و گستردگی پشتیبانی و جامعه متن‌باز آن است. هر دو چهارچوب به عنوان یک \lr{ORM}\LTRfootnote{Object-Relation Mapping} تعامل با پایگاه داده را بسیار ساده‌تر می‌کنند. به گونه‌ای که برای ارتباط با پایگاه‌داده به هیچ‌عنوان دغدغه‌ی نحوه تعریف روابط و ذخیره‌سازی اطلاعات را نداریم و فقط با مدل‌های تعریف شده‌ی داخل برنامه تعامل برقرار می‌کنیم. در کد این پروژه، ارتباطات با پایگاه داده جهت دریافت و ثبت اطلاعات در یک لایه‌ی کاملا مجزا با استفاده از الگوی شیئ دسترسی داده\LTRfootnote{Data Access Object} پیاده‌سازی شده است. این الگوی طراحی به نویسندگان کد این امکان را می‌‌دهد، که توابع و نیازمندی‌های دسترسی به داده ذخیره‌شده را در یک لایه‌ی کاملا جداگانه تعریف کنند و لایه‌های دخل و تصرف داده را در کنار این لایه توسعه دهند\cite{oracle_dao}.


با توجه به استفاده از الگوی تک‌مخزنه در نگهداری کد‌های پروژه، تا حد ممکن کد‌ها و منابع مشترک پروژه در پوشه \lr{pkg} نگهداری می‌شود. 

\subsection{قفل}
میکروسرویس مخصوص احراز‌ هویت و تعیین سطح دسترسی در این سامانه، قفل نام دارد. دلیل نام‌گذاری این میکروسرویس به این اسم، شباهت عملکرد قفل و این سامانه است که هر دو وظیفه کنترل دسترسی به منابع پشت قفل را دارند.

\begin{figure}
	\centering
	\includegraphics[scale=0.5]{figures/30bird-ghofl.jpg}
	\caption{عملکرد سرویس قفل}
	\label{fig:30bird-ghofl}
\end{figure}

در شکل\ref{fig:30bird-ghofl}، کاربرد سرویس قفل در سامانه مشخص شده‌است. \textbf{تمامی درخواست‌های ورودی} به سامانه در ابتدا به سرویس قفل فرستاده می‌شوند و احراز هویت آن‌ها انجام می‌شود. این درخواست‌ها یا مانند درخواست ورود و ثبت‌نام ماهیت احراز هویت دارند و یا سایر درخواست‌های کاربری و مدیریتی هستند. دسته اول درخواست‌ها مستقیما به قفل فرستاده می‌شوند. دسته دوم درخواست‌ها نیز هنگام رسیدن به دروازه ‌ورود، ابتدا به قفل فرستاده می‌شوند که سرایند احراز هویت \lr{JWT}\LTRfootnote{JSON Web Token} آنها ارزیابی شود. پس از این ارزیابی، در صورت موفقیت آمیز‌بودن. درخواست به سرویس‌های بالادستی فرستاده شده و درصورت ورود ناموفق کاربر با خطا مواجه خواهد شد.


\subsection{ناظم}
میکروسرویس ناظم وظیفه نظارت و مدیریت امور مالی و گزارشات مصرف کاربران را برعهده دارد. این میکروسرویس یکی از کلیدی‌ترین سرویس‌های سامانه است که در تعامل زیاد با پایگاه داده است. معماری کلی این سرویس در شکل\ref{fig:30bird-nazem} مشخص شده.

ناظم در \textbf{تعامل با باجه} در جایگاه یک اعتبارسنج\LTRfootnote{Validator} درخواست عمل می‌کند. به این صورت که تمامی درخواست‌های مربوط به دخل و تصرف در منابع باید ابتدا از این سامانه تاییدیه بگیرند. گرفتن این تاییدیه به معنی اعتبار کافی و نبودن محدودیت بر روی کاربر برای منابع درخواستی است. هرگونه ایراد و خطا در این سرویس منجر توقف عملکرد کل سامانه است که این امر یک موضوع پیش‌بینی شده و قابل قبول است. کد‌های مربوط به این اعتبار سنجی داخل بسته \lr{validation} قرار گرفته و برای توسعه و تغییر این قوانین باید این بسته دچار تغییر شود.

در تعامل با ‌دروازه‌ی ورود، درخواست‌های مدیر سامانه جهت تغییر محدودیت‌های کاربران یا گزارش گیری از وضعیت آنان را دریافت می‌کند و پاسخ مناسب را با توجه به اطلاعات ثبت شده در پایگاه داده می‌دهد.

این سرویس برای درخواست‌های دریافتی از سرویس خادم، گزارش مصرف منابع را برای منابع دریافت می‌کند و با توجه به قوانین و قیمت گذاری‌های ثبت‌شده، اعتبار کاربر را تحت تاثیر قرار می‌دهد.

\begin{figure}
	\vspace{1cm}
	\centering
	\includegraphics[scale=0.7]{figures/30bird-nazem.jpg}
	\caption{عملکرد سرویس ناظم}
	\label{fig:30bird-nazem}
\end{figure}


\subsection{باجه}
این سرویس دروازه ارتباطی کاربران عادی سامانه است. تمامی درخواست‌ها و عملیات‌های ممکن برای کاربر در این برنامه قرار گرفته. معماری این برنامه در شکل\ref{fig:30bird-baaje} مشخص شده است. همانطور که در تصویر مشخص شده، این سرویس مستقیما از طریق ‌دروازه‌ی ورود با کاربران در ارتباط است و برای انجام امور مربوط به کاربران با سرویس‌های خادم و ناظم به صورت داخلی در ارتباط است.


این سرویس اطلاعات مربوط به منابع تعریف شده برای کاربر را داخل پایگاه داده ذخیره کرده و در پردازش درخواست‌ها از این اطلاعات استفاده می‌کند. منابع تعریف شده داخل پایگاه داده شده داخل پیوست این گزارش موجود است.

در مواجه با درخواست‌های مربوط به تغییر منابع مانند ساخت یا ویرایش ماشین مجازی، ابتدا اعتبارسنجی درخواست از طریق سرویس ناظم انجام می‌شود و در صورت موفقیت‌آمیز بودن این اعتبارسنجی، درخواست جهت اجرا به سرویس خادم ارسال می‌شود. پس از رسیدن جواب از سمت سرویس خادم، بسته به خروجی سرویس خادم، نتیجه درخواست کاربر داخل پایگاه داده ذخیره می‌شود.
\begin{figure}
	\centering
	\includegraphics[scale=0.5]{figures/30bird-baaje.jpg}
	\caption{عملکرد سرویس باجه}
	\label{fig:30bird-baaje}
\end{figure}


\subsection{خادم}
همانطور که قبلا گفته شد، سامانه برای ارائه خدمات \lr{IaaS} نیازمند تعامل با یک ارائه دهنده زیرساخت است. این ارائه دهنده در سامانه‌ما \lr{VMWare Cloud Director} است، میکروسرویس مسئول جهت تعامل با این برنامه خادم است. معماری خادم از دو قسمت اصلی تشکیل شده است. \textbf{خدمت‌گزار} و \textbf{زمان‌بند}\LTRfootnote{Scheduler} دو بخش تشکیل دهنده این سرویس می‌باشند که کاملا مستقل و جداگانه اجرا می‌شوند. وظیفه خدمت‌گزار دریافت دستورات از سرویس باجه و سپس تفسیر و اجرای آن‌‌ها است. معماری کلی این پروژه در شکل \ref{fig:30bird-khadem}مشخص شده است.

پایه کد این پروژه به طور کلی در بسته \lr{vcloud} قرار داده شده. این بسته شامل تمامی عملیات‌های قابل اجرا بر روی \lr{cloud director} است. در صورت نیاز به تکمیل و یا توسعه خدمات بایستی دستورات و توابع این بسته را تغییر داد. داخل این بسته از بسته توسعه نرم‌افزار (\lr{SDK}\LTRfootnote{Software Development Kit}) رسمی \lr{VMWare Cloud Director} استفاده شده‌است. البته برای کاربرد‌های خاص و عملیات‌های تخصصی دقیق، ممکن است که نیاز به تعامل مستقیم با \lr{API} داشته باشیم. ولی برای خدمات فعلی سامانه، تمامی امکانات مورد نیاز داخل \lr{SDK} موجود است.

همچنین یک ثبت کننده وقایع داخل این سرویس تعبیه شده که تمامی درخواست‌های ارسالی به سرویس‌ بالا‌دستی را در یک پایگاه داده غیر رابطه‌ای ذخیره کند. این کار برای نظارت بیشتر و بازیابی از حادثه\LTRfootnote{Incident Recovery} احتمالی بسیار مفید است.

\begin{figure}
	\centering
	\includegraphics[scale=0.5]{figures/30bird-khadem.jpeg}
	\caption{عملکرد سرویس خادم}
	\label{fig:30bird-khadem}
\end{figure}

برنامه \textbf{زمان‌بند} در پس‌زمینه به صورت مستقل اجرا شده و در فواصل زمانی مشخص وضعیت منابع تعریف شده داخل \lr{Cloud Director} را رصد کرده و گزارش وضعیت را به سرویس ناظم جهت رسیدگی به امور مالی و نظارتی ارسال می‌کند. یک محدودیت مهم این سرویس، تک نسخه ای بودن آن در زمان اجرا است. برای قابلیت چند نسخه‌ای بودن این برنامه، بایستی یک مکانیزم و الگوریتم هماهنگی میان عناصر درحال اجرا پیاده‌سازی کنیم که منابع سیستم در هر لحظه تنها \textbf{توسط فقط یک نسخه} رصد و گزارش شود. پیاده‌سازی این موضوع با کمک امکانات کوبرنتیز امکان پذیر است ولی بخاطر پیچیدگی‌هایی که این پیاده‌سازی به همراه دارد در این پروژه از این‌کار صرف نظر شده‌است.

برنامه خدمت‌گزار در پس‌زمینه اجرا می‌شود و منتظر دستورات از سمت باجه می‌شود. پس از رسیدن دستورات، بسته به نوع دستور، توابع و پیاده‌سازی‌های موجود در بسته \lr{vcloud} در برنامه اجرا می‌شود. با این کار، دستورات قابل فهم توسط \lr{cloud director} ارسال شده و نتیجه به صورت برخط و یا غیر برخط به کاربر گزارش می‌شود.


\section{‌دروازه‌ی ورود رابط}
پیاده‌سازی ‌دروازه‌ی ورود رابط در این سامانه با استفاده از ابزار قدرتمند و پایدار \lr{nginx} صورت گرفته. \lr{nginx} یک خدمت‌گزار وب نوشته شده با زبان \lr{C} است که توانایی مدیریت تعداد بسیار زیادی از درخواست‌ها در لحظه را دارد. برای استفاده از این ‌دروازه‌ی ورود باید قوانین مربوط به مسیریابی را در فرمت مخصوص بنویسیم. از آنجایی که مقیاس کردن این گزینه به صورت افقی سخت انجام می‌شود (باید قوانین مربوط به \lr{DNS} و کاربران را تغییر دهیم.) بایستی منابع و ظرفیت‌های این برنامه‌ را زیاد‌تر از سایر قسمت‌های دیگر در نظر بگیریم.



\section{نتیجه گیری}
در این فصل به تشریح قسمت‌های مختلف برنامه و جزئیات پیاده‌سازی آن‌ها پرداخته شد. سامانه ما از چهار میکروسرویس که هریک وظیفه مخصوص به خود را دارد تشکیل شد. این میکروسرویس‌ها که در قالب بسته‌های داکر اجرا شدند با رابط‌های \lr{REST} به یکدیگر متصل هستند. همه این میکروسرویس‌ها در پشت یک ‌دروازه‌ی ورود قرار گرفته و راه ارتباطی با شبکه خصوصی سامانه تنها این ‌دروازه‌ی ورود است. اولین میکروسرویس درگیر با درخواست‌های کاربران، \textbf{قفل} هست که وظیفه احراز هویت تمامی درخواست‌های ورودی به ‌دروازه‌ی ورود را دارد. میکروسرویس بعدی، \textbf{باجه} است که تمامی عملیات‌های کاربران در آن تعریف شد. این میکروسرویس برای صحت سنجی درخواست‌های کاربران با میکروسرویس \textbf{ناظم} در ارتباط است و برای انجام این درخواست‌ها با میکروسرویس \textbf{خادم} ارتباط برقرار می‌کند.

